

%In order to make use of the compliment rule we must first verify that our three events are, indeed, complimentary. To do this, we must show two things:

%First, we must show that our three events are disjoint, or mutually exclusive. Meaning that two of the events cannot happen simultaneously. This is true intuitively, if a fusion is between two sex chromosomes, \begin{math}SS \end{math}, then it cannot simultaneously be a fusion between two autosomes, \begin{math}AA \end{math}. The other combinations of events, \begin{math}SS \mbox{ and }AS \end{math} fusion etc., are also mutually exclusive by the same logic.

%Second, we must show that our three events exhaust the sample space. This is, again, intuitively true. If a fusion occurs between two chromosomes, each chromosome can either be a sex chromosome or an autosome. So the three possible fusions are when both are sex chromosomes or both autosomes and when one is an autosome and one is a sex chromosome. 

